% Options for packages loaded elsewhere
\PassOptionsToPackage{unicode}{hyperref}
\PassOptionsToPackage{hyphens}{url}
%
\documentclass[
]{article}
\usepackage{amsmath,amssymb}
\usepackage{iftex}
\ifPDFTeX
  \usepackage[T1]{fontenc}
  \usepackage[utf8]{inputenc}
  \usepackage{textcomp} % provide euro and other symbols
\else % if luatex or xetex
  \usepackage{unicode-math} % this also loads fontspec
  \defaultfontfeatures{Scale=MatchLowercase}
  \defaultfontfeatures[\rmfamily]{Ligatures=TeX,Scale=1}
\fi
\usepackage{lmodern}
\ifPDFTeX\else
  % xetex/luatex font selection
\fi
% Use upquote if available, for straight quotes in verbatim environments
\IfFileExists{upquote.sty}{\usepackage{upquote}}{}
\IfFileExists{microtype.sty}{% use microtype if available
  \usepackage[]{microtype}
  \UseMicrotypeSet[protrusion]{basicmath} % disable protrusion for tt fonts
}{}
\makeatletter
\@ifundefined{KOMAClassName}{% if non-KOMA class
  \IfFileExists{parskip.sty}{%
    \usepackage{parskip}
  }{% else
    \setlength{\parindent}{0pt}
    \setlength{\parskip}{6pt plus 2pt minus 1pt}}
}{% if KOMA class
  \KOMAoptions{parskip=half}}
\makeatother
\usepackage{xcolor}
\usepackage[margin=1in]{geometry}
\usepackage{color}
\usepackage{fancyvrb}
\newcommand{\VerbBar}{|}
\newcommand{\VERB}{\Verb[commandchars=\\\{\}]}
\DefineVerbatimEnvironment{Highlighting}{Verbatim}{commandchars=\\\{\}}
% Add ',fontsize=\small' for more characters per line
\usepackage{framed}
\definecolor{shadecolor}{RGB}{248,248,248}
\newenvironment{Shaded}{\begin{snugshade}}{\end{snugshade}}
\newcommand{\AlertTok}[1]{\textcolor[rgb]{0.94,0.16,0.16}{#1}}
\newcommand{\AnnotationTok}[1]{\textcolor[rgb]{0.56,0.35,0.01}{\textbf{\textit{#1}}}}
\newcommand{\AttributeTok}[1]{\textcolor[rgb]{0.13,0.29,0.53}{#1}}
\newcommand{\BaseNTok}[1]{\textcolor[rgb]{0.00,0.00,0.81}{#1}}
\newcommand{\BuiltInTok}[1]{#1}
\newcommand{\CharTok}[1]{\textcolor[rgb]{0.31,0.60,0.02}{#1}}
\newcommand{\CommentTok}[1]{\textcolor[rgb]{0.56,0.35,0.01}{\textit{#1}}}
\newcommand{\CommentVarTok}[1]{\textcolor[rgb]{0.56,0.35,0.01}{\textbf{\textit{#1}}}}
\newcommand{\ConstantTok}[1]{\textcolor[rgb]{0.56,0.35,0.01}{#1}}
\newcommand{\ControlFlowTok}[1]{\textcolor[rgb]{0.13,0.29,0.53}{\textbf{#1}}}
\newcommand{\DataTypeTok}[1]{\textcolor[rgb]{0.13,0.29,0.53}{#1}}
\newcommand{\DecValTok}[1]{\textcolor[rgb]{0.00,0.00,0.81}{#1}}
\newcommand{\DocumentationTok}[1]{\textcolor[rgb]{0.56,0.35,0.01}{\textbf{\textit{#1}}}}
\newcommand{\ErrorTok}[1]{\textcolor[rgb]{0.64,0.00,0.00}{\textbf{#1}}}
\newcommand{\ExtensionTok}[1]{#1}
\newcommand{\FloatTok}[1]{\textcolor[rgb]{0.00,0.00,0.81}{#1}}
\newcommand{\FunctionTok}[1]{\textcolor[rgb]{0.13,0.29,0.53}{\textbf{#1}}}
\newcommand{\ImportTok}[1]{#1}
\newcommand{\InformationTok}[1]{\textcolor[rgb]{0.56,0.35,0.01}{\textbf{\textit{#1}}}}
\newcommand{\KeywordTok}[1]{\textcolor[rgb]{0.13,0.29,0.53}{\textbf{#1}}}
\newcommand{\NormalTok}[1]{#1}
\newcommand{\OperatorTok}[1]{\textcolor[rgb]{0.81,0.36,0.00}{\textbf{#1}}}
\newcommand{\OtherTok}[1]{\textcolor[rgb]{0.56,0.35,0.01}{#1}}
\newcommand{\PreprocessorTok}[1]{\textcolor[rgb]{0.56,0.35,0.01}{\textit{#1}}}
\newcommand{\RegionMarkerTok}[1]{#1}
\newcommand{\SpecialCharTok}[1]{\textcolor[rgb]{0.81,0.36,0.00}{\textbf{#1}}}
\newcommand{\SpecialStringTok}[1]{\textcolor[rgb]{0.31,0.60,0.02}{#1}}
\newcommand{\StringTok}[1]{\textcolor[rgb]{0.31,0.60,0.02}{#1}}
\newcommand{\VariableTok}[1]{\textcolor[rgb]{0.00,0.00,0.00}{#1}}
\newcommand{\VerbatimStringTok}[1]{\textcolor[rgb]{0.31,0.60,0.02}{#1}}
\newcommand{\WarningTok}[1]{\textcolor[rgb]{0.56,0.35,0.01}{\textbf{\textit{#1}}}}
\usepackage{graphicx}
\makeatletter
\def\maxwidth{\ifdim\Gin@nat@width>\linewidth\linewidth\else\Gin@nat@width\fi}
\def\maxheight{\ifdim\Gin@nat@height>\textheight\textheight\else\Gin@nat@height\fi}
\makeatother
% Scale images if necessary, so that they will not overflow the page
% margins by default, and it is still possible to overwrite the defaults
% using explicit options in \includegraphics[width, height, ...]{}
\setkeys{Gin}{width=\maxwidth,height=\maxheight,keepaspectratio}
% Set default figure placement to htbp
\makeatletter
\def\fps@figure{htbp}
\makeatother
\setlength{\emergencystretch}{3em} % prevent overfull lines
\providecommand{\tightlist}{%
  \setlength{\itemsep}{0pt}\setlength{\parskip}{0pt}}
\setcounter{secnumdepth}{-\maxdimen} % remove section numbering
\ifLuaTeX
  \usepackage{selnolig}  % disable illegal ligatures
\fi
\usepackage{bookmark}
\IfFileExists{xurl.sty}{\usepackage{xurl}}{} % add URL line breaks if available
\urlstyle{same}
\hypersetup{
  pdftitle={M5\_AI3\_CANOJORGE},
  pdfauthor={JORGE CANO},
  hidelinks,
  pdfcreator={LaTeX via pandoc}}

\title{M5\_AI3\_CANOJORGE}
\author{JORGE CANO}
\date{2024-10-20}

\begin{document}
\maketitle

\subsection{Utilizando la base de datos de pisos, que hemos utilizado
durante el temario en la que podemos encontrar un listado de pisos
disponibles en Airbnb en Madrid, por temas computacionales, debes
quedarte con un máximo de 2000 viviendas para responder las siguientes
preguntas:}\label{utilizando-la-base-de-datos-de-pisos-que-hemos-utilizado-durante-el-temario-en-la-que-podemos-encontrar-un-listado-de-pisos-disponibles-en-airbnb-en-madrid-por-temas-computacionales-debes-quedarte-con-un-muxe1ximo-de-2000-viviendas-para-responder-las-siguientes-preguntas}

\section{Lo primero será traer los datos y limitar la BBDD a 2.000
observaciones.}\label{lo-primero-seruxe1-traer-los-datos-y-limitar-la-bbdd-a-2.000-observaciones.}

\begin{verbatim}
##      X longitude latitude  price       room_type minimum_nights
## 1 2948  -3.58253 40.47534  57.00 Entire home/apt              4
## 2 6990  -3.70754 40.45151 178.00 Entire home/apt              1
## 3 7784  -3.70115 40.47011  69.00 Entire home/apt              3
## 4  645  -3.76036 40.40210 125.00 Entire home/apt              1
## 5 3250  -3.69328 40.40821  69.86 Entire home/apt              1
## 6 2460  -3.70531 40.42623 110.00 Entire home/apt              2
##   number_of_reviews review_scores_value calculated_host_listings_count bedrooms
## 1               379                   9                              3        2
## 2                 1                   8                              4        3
## 3                 1                   8                              1        3
## 4                20                   9                              6        1
## 5               228                  10                              3        1
## 6               107                   9                              1        2
##   reviews_per_month beds accommodates availability_30 availability_60
## 1              9.90    5            4              29              59
## 2              0.12    4            6              30              60
## 3              0.73    3            5              29              59
## 4              0.29    4            6              20              46
## 5              6.57    3            3               9              24
## 6              2.59    4            6              13              43
##   availability_90 instant_bookable Distancia_Centro Distancia_Norte
## 1              89                t             12.1             8.5
## 2              90                f              3.9             3.1
## 3              89                t              5.9             1.6
## 4              76                t              5.1            10.2
## 5              48                t              1.3             7.1
## 6              73                t              1.1             5.4
##   Distancia_Sur logprice tv_ports phone_ports Vecinos Piso ventanas
## 1          15.2 4.043051        4           3       2    5        2
## 2           6.9 5.181784        1           2       1    6        6
## 3           9.0 4.234107        3           2       2    4        3
## 4           3.3 4.828314        1           4       1    5        2
## 5           3.3 4.246493        3           2       1    1        1
## 6           4.2 4.700480        1           4       2    1        4
\end{verbatim}

\begin{verbatim}
## 'data.frame':    2000 obs. of  26 variables:
##  $ X                             : int  2948 6990 7784 645 3250 2460 1644 3212 6248 7484 ...
##  $ longitude                     : num  -3.58 -3.71 -3.7 -3.76 -3.69 ...
##  $ latitude                      : num  40.5 40.5 40.5 40.4 40.4 ...
##  $ price                         : num  57 178 69 125 69.9 ...
##  $ room_type                     : chr  "Entire home/apt" "Entire home/apt" "Entire home/apt" "Entire home/apt" ...
##  $ minimum_nights                : int  4 1 3 1 1 2 1 2 2 1 ...
##  $ number_of_reviews             : int  379 1 1 20 228 107 2 5 8 1 ...
##  $ review_scores_value           : int  9 8 8 9 10 9 8 9 7 2 ...
##  $ calculated_host_listings_count: int  3 4 1 6 3 1 20 1 4 1 ...
##  $ bedrooms                      : int  2 3 3 1 1 2 1 1 2 1 ...
##  $ reviews_per_month             : num  9.9 0.12 0.73 0.29 6.57 2.59 0.04 0.14 0.71 0.29 ...
##  $ beds                          : int  5 4 3 4 3 4 3 2 2 4 ...
##  $ accommodates                  : int  4 6 5 6 3 6 4 4 6 6 ...
##  $ availability_30               : int  29 30 29 20 9 13 0 0 30 0 ...
##  $ availability_60               : int  59 60 59 46 24 43 0 0 60 0 ...
##  $ availability_90               : int  89 90 89 76 48 73 0 0 88 0 ...
##  $ instant_bookable              : chr  "t" "f" "t" "t" ...
##  $ Distancia_Centro              : num  12.1 3.9 5.9 5.1 1.3 1.1 0.8 0.9 3.5 7.2 ...
##  $ Distancia_Norte               : num  8.5 3.1 1.6 10.2 7.1 5.4 7.2 7.3 9.9 13.2 ...
##  $ Distancia_Sur                 : num  15.2 6.9 9 3.3 3.3 4.2 2.5 2.4 1 5.3 ...
##  $ logprice                      : num  4.04 5.18 4.23 4.83 4.25 ...
##  $ tv_ports                      : int  4 1 3 1 3 1 2 2 4 3 ...
##  $ phone_ports                   : int  3 2 2 4 2 4 3 2 4 2 ...
##  $ Vecinos                       : int  2 1 2 1 1 2 1 1 3 1 ...
##  $ Piso                          : int  5 6 4 5 1 1 4 5 6 2 ...
##  $ ventanas                      : int  2 6 3 2 1 4 4 4 3 2 ...
\end{verbatim}

\begin{verbatim}
##        X          longitude         latitude         price       
##  Min.   :   3   Min.   :-3.794   Min.   :40.33   Min.   : 20.00  
##  1st Qu.:1886   1st Qu.:-3.707   1st Qu.:40.41   1st Qu.: 56.00  
##  Median :3737   Median :-3.702   Median :40.42   Median : 76.50  
##  Mean   :3845   Mean   :-3.698   Mean   :40.42   Mean   : 97.78  
##  3rd Qu.:5809   3rd Qu.:-3.695   3rd Qu.:40.43   3rd Qu.:110.25  
##  Max.   :7899   Max.   :-3.576   Max.   :40.51   Max.   :847.00  
##   room_type         minimum_nights  number_of_reviews review_scores_value
##  Length:2000        Min.   :1.000   Min.   :  1.0     Min.   : 2.000     
##  Class :character   1st Qu.:1.000   1st Qu.:  6.0     1st Qu.: 9.000     
##  Mode  :character   Median :2.000   Median : 23.5     Median : 9.000     
##                     Mean   :2.296   Mean   : 57.3     Mean   : 9.185     
##                     3rd Qu.:3.000   3rd Qu.: 77.0     3rd Qu.:10.000     
##                     Max.   :9.000   Max.   :643.0     Max.   :10.000     
##  calculated_host_listings_count    bedrooms    reviews_per_month
##  Min.   :  1.00                 Min.   :1.00   Min.   :0.010    
##  1st Qu.:  1.00                 1st Qu.:1.00   1st Qu.:0.340    
##  Median :  2.00                 Median :1.00   Median :1.070    
##  Mean   : 13.59                 Mean   :1.69   Mean   :1.594    
##  3rd Qu.:  7.00                 3rd Qu.:2.00   3rd Qu.:2.360    
##  Max.   :213.00                 Max.   :9.00   Max.   :9.900    
##       beds         accommodates    availability_30 availability_60
##  Min.   : 0.000   Min.   : 1.000   Min.   : 0.00   Min.   : 0.00  
##  1st Qu.: 2.000   1st Qu.: 3.000   1st Qu.: 0.00   1st Qu.: 0.00  
##  Median : 2.000   Median : 4.000   Median :14.00   Median :39.00  
##  Mean   : 2.532   Mean   : 4.381   Mean   :13.93   Mean   :30.68  
##  3rd Qu.: 3.000   3rd Qu.: 6.000   3rd Qu.:29.00   3rd Qu.:59.00  
##  Max.   :14.000   Max.   :16.000   Max.   :30.00   Max.   :60.00  
##  availability_90 instant_bookable   Distancia_Centro Distancia_Norte 
##  Min.   : 0.00   Length:2000        Min.   : 0.0     Min.   : 0.100  
##  1st Qu.: 0.00   Class :character   1st Qu.: 0.7     1st Qu.: 5.400  
##  Median :65.00   Mode  :character   Median : 1.0     Median : 6.400  
##  Mean   :48.31                      Mean   : 1.9     Mean   : 6.185  
##  3rd Qu.:88.00                      3rd Qu.: 2.5     3rd Qu.: 7.100  
##  Max.   :90.00                      Max.   :12.6     Max.   :15.700  
##  Distancia_Sur       logprice        tv_ports     phone_ports      Vecinos     
##  Min.   : 0.200   Min.   :2.996   Min.   :1.00   Min.   :1.00   Min.   :1.000  
##  1st Qu.: 2.900   1st Qu.:4.025   1st Qu.:1.00   1st Qu.:2.00   1st Qu.:1.000  
##  Median : 3.650   Median :4.337   Median :2.00   Median :3.00   Median :2.000  
##  Mean   : 4.287   Mean   :4.402   Mean   :2.48   Mean   :2.51   Mean   :1.963  
##  3rd Qu.: 4.800   3rd Qu.:4.703   3rd Qu.:4.00   3rd Qu.:3.00   3rd Qu.:3.000  
##  Max.   :15.700   Max.   :6.742   Max.   :4.00   Max.   :4.00   Max.   :3.000  
##       Piso        ventanas    
##  Min.   :1.0   Min.   : 1.00  
##  1st Qu.:2.0   1st Qu.: 2.00  
##  Median :4.0   Median : 3.00  
##  Mean   :3.5   Mean   : 3.17  
##  3rd Qu.:5.0   3rd Qu.: 4.00  
##  Max.   :6.0   Max.   :11.00
\end{verbatim}

\subsection{1. ¿Existe dependencia espacial en la variable precio? ¿Qué
tipo de dependencia espacial existe: local, global o
ambas?}\label{existe-dependencia-espacial-en-la-variable-precio-quuxe9-tipo-de-dependencia-espacial-existe-local-global-o-ambas}

Para contestar a esta pregunta vamos a calcular los k-nearest neighbors
(k = 10) y posteriormente realizar un tes de dependencia espacial con el
test de Moran.

El índice de Moran es una medida de autocorrelación espacial global. Un
valor significativo indica que existe dependencia espacial global, es
decir, los valores de price están correlacionados espacialmente en toda
el área de estudio.

Si el p-valor es significativo (menor a 0.05), indica que existe una
autocorrelación espacial global en la variable price.En nuestro caso
obtenemos un p valor inferior a 0.05 y por tanto \textbf{podemos
determinar que existe autocorrelación espacial global en la variable
precio}

\begin{Shaded}
\begin{Highlighting}[]
\NormalTok{nb }\OtherTok{\textless{}{-}} \FunctionTok{knn2nb}\NormalTok{(}\FunctionTok{knearneigh}\NormalTok{(}\FunctionTok{cbind}\NormalTok{(BBDD\_MUESTRA}\SpecialCharTok{$}\NormalTok{longitude, BBDD\_MUESTRA}\SpecialCharTok{$}\NormalTok{latitude), }\AttributeTok{k=}\DecValTok{10}\NormalTok{))}


\FunctionTok{moran.test}\NormalTok{(}\AttributeTok{x =}\NormalTok{ BBDD\_MUESTRA}\SpecialCharTok{$}\NormalTok{price, }\AttributeTok{listw =} \FunctionTok{nb2listw}\NormalTok{(nb, }\AttributeTok{style=}\StringTok{"W"}\NormalTok{))}
\end{Highlighting}
\end{Shaded}

\begin{verbatim}
Moran I test under randomisation
\end{verbatim}

data: BBDD\_MUESTRA\$price\\
weights: nb2listw(nb, style = ``W'')

Moran I statistic standard deviate = 4.9772, p-value = 3.226e-07
alternative hypothesis: greater sample estimates: Moran I statistic
Expectation Variance 4.614511e-02 -5.002501e-04 8.783103e-05

\begin{Shaded}
\begin{Highlighting}[]
\FunctionTok{moran.plot}\NormalTok{(}\AttributeTok{x =}\NormalTok{ BBDD\_MUESTRA}\SpecialCharTok{$}\NormalTok{price, }\AttributeTok{listw =} \FunctionTok{nb2listw}\NormalTok{(nb, }\AttributeTok{style=}\StringTok{"W"}\NormalTok{),}\AttributeTok{main=}\StringTok{"Gráfico I Moran"}\NormalTok{)}
\end{Highlighting}
\end{Shaded}

\includegraphics{M5_AI3_CANOJORGE_files/figure-latex/unnamed-chunk-3-1.pdf}
Para ver si además de esto, hay alguna zona en el mapa que presenta un
alto grado de dependencia espacial. Es decir, una dependencia espacial
local, podemos llamar al test LISA. Es equivalente al I-Moran pero lo
vamos a hacer a nivel regiones.

\begin{verbatim}
##            Ii          E.Ii      Var.Ii       Z.Ii Pr(z != E(Ii))
## 1  0.20599534 -1.334961e-04 0.026575399  1.2644418      0.2060715
## 2 -0.23660622 -5.167205e-04 0.102825418 -0.7362519      0.4615774
## 3  0.10242141 -6.648497e-05 0.013236217  0.8908214      0.3730250
## 4 -0.03891009 -5.950355e-05 0.011846399 -0.3569476      0.7211310
## 5  0.08073519 -6.257046e-05 0.012456943  0.7239250      0.4691118
## 6 -0.03055838 -1.199640e-05 0.002388443 -0.6250321      0.5319500
\end{verbatim}

Representando el test gráficamente, \textbf{podemos confirmar que existe
dependencia local}

\begin{verbatim}
## Error: object 'lw' not found
\end{verbatim}

\begin{verbatim}
## Error: object 'lisa' not found
\end{verbatim}

\begin{verbatim}
## Error: object 'lisa' not found
\end{verbatim}

\begin{verbatim}
## Error in `geom_point()`:
## ! Problem while computing aesthetics.
## i Error occurred in the 1st layer.
## Caused by error:
## ! object 'lisa_value' not found
\end{verbatim}

\subsection{2. Establece un modelo lineal para estimar la variable
precio por m2. ¿Hay dependencia espacial en los residuos del
modelo?}\label{establece-un-modelo-lineal-para-estimar-la-variable-precio-por-m2.-hay-dependencia-espacial-en-los-residuos-del-modelo}

A la hora de crear un modelo líneal utilizaremos el siguiente modelo
glm: price \textasciitilde{} minimum\_nights + number\_of\_reviews +
beds + Distancia\_Centro + Piso + ventanas

Dado que no me ejecute la función \textbf{lagsarlm} vamos a trabajar
bajo la siguiente hipótesis respondiende de una forma teórica:

\textbf{comenta el problema como si Rho y Lambda fuesen significativos,
qué significaría, cómo impactaría}

En el contexto de un modelo de regresión espacial autorregresivo (SAR),
Rho (ρ) es el parámetro que mide la autocorrelación espacial en la
variable dependiente (Y). \textbf{Si Rho es significativo, esto implica
que los valores de la variable dependiente en una localización están
influenciados por los valores de la variable dependiente en
localizaciones vecinas.}

Si ρ es positivo, esto sugiere que hay dependencia positiva, es decir,
las observaciones con valores altos de la variable dependiente tienden a
estar cerca de otras observaciones con valores altos, y lo mismo ocurre
con valores bajos. Si ρ es negativo, esto indicaría dependencia
negativa, lo que significa que las observaciones con valores altos de la
variable dependiente tienden a estar cerca de observaciones con valores
bajos, lo que sugiere dispersión espacial.

En un modelo de error espacial (SEM), Lambda (λ) representa la
autocorrelación espacial en los errores del modelo. \textbf{Si Lambda es
significativo, esto sugiere que hay un patrón espacial en los residuos
(es decir, los errores no son independientes entre sí, sino que están
correlacionados espacialmente).}

\begin{verbatim}
## Error in lagsarlm(formula = formula, data = BBDD_MUESTRA, listw = nb2listw(nb3, : could not find function "lagsarlm"
\end{verbatim}

\begin{verbatim}
## Error: object 'modelo_espacial_sar' not found
\end{verbatim}

\begin{verbatim}
## [1] "residuos modelo GLM 10159684.8016581"
\end{verbatim}

\begin{verbatim}
## Error: object 'modelo_espacial_sar' not found
\end{verbatim}

\begin{Shaded}
\begin{Highlighting}[]
\CommentTok{\#install.packages(\textquotesingle{}spDataLarge\textquotesingle{}, repos=\textquotesingle{}https://nowosad.github.io/drat/\textquotesingle{}, type=\textquotesingle{}source\textquotesingle{})}
\end{Highlighting}
\end{Shaded}

\subsection{3. Introduce una variable más en el modelo. Dicha variable
es la distancia mínima entre cada persona y la geolocalización de las
oficinas bancarias de Madrid obtenidas con OSM. ¿Sigue habiendo
dependencia espacial en los residuos del nuevo
modelo?}\label{introduce-una-variable-muxe1s-en-el-modelo.-dicha-variable-es-la-distancia-muxednima-entre-cada-persona-y-la-geolocalizaciuxf3n-de-las-oficinas-bancarias-de-madrid-obtenidas-con-osm.-sigue-habiendo-dependencia-espacial-en-los-residuos-del-nuevo-modelo}

Para contestar el ejercicio nos traemos la BBDD de OSM. Una vez cargada
calculamos la variable de distancia mínima entre cada persona y la
geolocalización de las oficinas bancarias de Madrid obtenidas con OSM.

Posteriormente calculamos el p-valor del test de Moran, siendo este mur
bajo (\textless{} 0.05), lo que nos indica que sigue habiendo
autocorrelación espacial en los residuos del nuevo modelo.

\begin{verbatim}
##     CIA  LONG_IND  LAT_IND
## 1 Caixa -3.481729 40.53103
## 2 Caixa -3.989592 40.23070
## 3 Caixa -3.370189 40.50776
## 4 Caixa -3.376134 40.49007
## 5 Caixa -3.355582 40.48619
## 6 Caixa -3.376853 40.47518
\end{verbatim}

\begin{verbatim}
##      X longitude latitude  price       room_type minimum_nights
## 1 2948  -3.58253 40.47534  57.00 Entire home/apt              4
## 2 6990  -3.70754 40.45151 178.00 Entire home/apt              1
## 3 7784  -3.70115 40.47011  69.00 Entire home/apt              3
## 4  645  -3.76036 40.40210 125.00 Entire home/apt              1
## 5 3250  -3.69328 40.40821  69.86 Entire home/apt              1
## 6 2460  -3.70531 40.42623 110.00 Entire home/apt              2
##   number_of_reviews review_scores_value calculated_host_listings_count bedrooms
## 1               379                   9                              3        2
## 2                 1                   8                              4        3
## 3                 1                   8                              1        3
## 4                20                   9                              6        1
## 5               228                  10                              3        1
## 6               107                   9                              1        2
##   reviews_per_month beds accommodates availability_30 availability_60
## 1              9.90    5            4              29              59
## 2              0.12    4            6              30              60
## 3              0.73    3            5              29              59
## 4              0.29    4            6              20              46
## 5              6.57    3            3               9              24
## 6              2.59    4            6              13              43
##   availability_90 instant_bookable Distancia_Centro Distancia_Norte
## 1              89                t             12.1             8.5
## 2              90                f              3.9             3.1
## 3              89                t              5.9             1.6
## 4              76                t              5.1            10.2
## 5              48                t              1.3             7.1
## 6              73                t              1.1             5.4
##   Distancia_Sur logprice tv_ports phone_ports Vecinos Piso ventanas
## 1          15.2 4.043051        4           3       2    5        2
## 2           6.9 5.181784        1           2       1    6        6
## 3           9.0 4.234107        3           2       2    4        3
## 4           3.3 4.828314        1           4       1    5        2
## 5           3.3 4.246493        3           2       1    1        1
## 6           4.2 4.700480        1           4       2    1        4
##   distancia_a_agencia
## 1           474.00356
## 2           191.94819
## 3            89.82234
## 4           729.18021
## 5           191.70566
## 6           116.50945
\end{verbatim}

\begin{verbatim}
## NULL
\end{verbatim}

\begin{verbatim}
## 
## Call:
## glm(formula = formula_new_model, family = gaussian, data = BBDD_MUESTRA)
## 
## Coefficients:
##                     Estimate Std. Error t value Pr(>|t|)    
## (Intercept)        71.714083   7.177114   9.992  < 2e-16 ***
## minimum_nights     -2.075558   1.158992  -1.791 0.073473 .  
## number_of_reviews  -0.180852   0.020593  -8.782  < 2e-16 ***
## beds               15.727672   1.257459  12.508  < 2e-16 ***
## Distancia_Centro   -3.295501   0.860629  -3.829 0.000133 ***
## Piso2              -1.011702   5.575937  -0.181 0.856040    
## Piso3              -1.309096   5.656959  -0.231 0.817018    
## Piso4              -3.669315   5.487777  -0.669 0.503807    
## Piso5              -5.833487   5.588203  -1.044 0.296662    
## Piso6               0.899282   5.607629   0.160 0.872608    
## ventanas2           9.930774   5.737042   1.731 0.083609 .  
## ventanas3           8.683027   5.655661   1.535 0.124875    
## ventanas4           6.421872   5.639399   1.139 0.254945    
## ventanas5          15.930695   6.886222   2.313 0.020802 *  
## ventanas6          56.131284  11.181829   5.020 5.63e-07 ***
## ventanas7          23.930522  20.925902   1.144 0.252935    
## ventanas8          59.250014  37.476603   1.581 0.114041    
## ventanas9         180.642325  51.643896   3.498 0.000479 ***
## ventanas11        -85.818835  72.195513  -1.189 0.234700    
## distancia_minima   -0.002995   0.015934  -0.188 0.850929    
## ---
## Signif. codes:  0 '***' 0.001 '**' 0.01 '*' 0.05 '.' 0.1 ' ' 1
## 
## (Dispersion parameter for gaussian family taken to be 5131.062)
## 
##     Null deviance: 12461378  on 1999  degrees of freedom
## Residual deviance: 10159504  on 1980  degrees of freedom
## AIC: 22784
## 
## Number of Fisher Scoring iterations: 2
\end{verbatim}

\begin{verbatim}
## 
##  Moran I test under randomisation
## 
## data:  residuos  
## weights: listw    
## 
## Moran I statistic standard deviate = 3.2048, p-value = 0.0006758
## alternative hypothesis: greater
## sample estimates:
## Moran I statistic       Expectation          Variance 
##      2.950927e-02     -5.002501e-04      8.768297e-05
\end{verbatim}

\subsection{4. Modeliza el precio con un SAR. ¿Es significativo el
factor de dependencia espacial? Interpreta el
modelo.}\label{modeliza-el-precio-con-un-sar.-es-significativo-el-factor-de-dependencia-espacial-interpreta-el-modelo.}

Para contestar este ejercicio modelizo el precio con un SAR. Al igual
que ejercicios anteriores no puedo ejecutar la funcion lagsarlm a pesar
de tener instalado la librería spdep. Por ello, dejaré reflejado el
código, y contestaré bajo la hipótesis de que Rho y Lambda son igual
0.5.

\textbf{Rho = 0.5}: Indica que los precios de las viviendas están
moderadamente influenciados por los precios en las viviendas vecinas
(50\% de autocorrelación espacial en la variable dependiente).

\textbf{Lambda = 0.5}: Indica que los errores del modelo tienen un 50\%
de autocorrelación espacial, sugiriendo que factores espaciales no
modelados aún están afectando a los precios.

La presencia de ambos (Rho y Lambda) sugiere que la estructura espacial
es importante y que hay dependencia espacial tanto en los precios como
en los errores, lo que justifica el uso de un modelo SAR para obtener
estimaciones más precisas y robustas.

\begin{verbatim}
## Error in lagsarlm(formula = formula_sar, data = BBDD_MUESTRA, listw = listw): could not find function "lagsarlm"
\end{verbatim}

\begin{verbatim}
## Error: object 'modelo_sar' not found
\end{verbatim}

\subsection{5. Modeliza el precio con un SEM. ¿Es significativo el
factor de dependencia espacial? Interpreta el
modelo.}\label{modeliza-el-precio-con-un-sem.-es-significativo-el-factor-de-dependencia-espacial-interpreta-el-modelo.}

Debido a los problemas con la librería spdep tampoco puedo ejecutar la
función errorsarlm y por tanto no puedo modelizar el precio con un SEM,
por lo que aplicaré un modelo teorico. Podemos reusmir el modelo teórico
de SEM bajo las siguientes premisas:

\textbf{Coeficientes:} Indican cómo afectan las variables independientes
a la variable dependiente, ajustando por la dependencia espacial en los
errores. El tamaño y el signo de los coeficientes muestran la dirección
y magnitud del impacto de cada variable.

\textbf{Lambda (λ):} Mide la autocorrelación espacial en los errores. Si
λ es significativo, hay dependencia espacial en los errores, lo que
significa que el modelo SEM es adecuado. El valor de λ indica el grado
de dependencia espacial en los errores.

\textbf{LR test:} Un p-valor bajo en el LR test indica que el modelo SEM
es significativamente mejor que un modelo sin autocorrelación espacial
en los errores.

\textbf{AIC:} El AIC permite comparar la calidad del ajuste entre
diferentes modelos. Un valor más bajo indica un mejor ajuste.

Para el ejercicio propuesto vamos a suponer un p valor de significacia
de los coeficientes inferior a 0.05, representando que es
estadísticamente significativo, lo que indica que la variable
independiente tiene un impacto en la variable dependiente después de
ajustar por la dependencia espacial en los errores.

Si λ es significativo (p-valor \textless{} 0.05), indica que existe
autocorrelación espacial en los errores. Los errores no son
independientes y están correlacionados en función de la localización
espacial. Esto sugiere que hay factores espaciales no modelados que
están afectando las observaciones.

\begin{verbatim}
## Error in errorsarlm(formula = formula, data = BBDD_MUESTRA, listw = nb2listw(nb3, : could not find function "errorsarlm"
\end{verbatim}

\begin{verbatim}
## Error: object 'modelo_espacial_sem' not found
\end{verbatim}

\subsection{6. Valora la capacidad predictiva del modelo SAR con la
técnica de validación
cruzada.}\label{valora-la-capacidad-predictiva-del-modelo-sar-con-la-tuxe9cnica-de-validaciuxf3n-cruzada.}

\subsection{7. Propón un modelo GWR para estimar los residuos con un
cierto
suavizado.}\label{propuxf3n-un-modelo-gwr-para-estimar-los-residuos-con-un-cierto-suavizado.}

Un modelo GWR es útil cuando queremos modelar una relación que varía en
el espacio. Este modelo permite que los coeficientes de las variables
explicativas varíen dependiendo de la localización, lo que lo hace
especialmente útil en situaciones en las que las relaciones entre las
variables no son homogéneas en todas las regiones. La GWR se ajusta a
los datos ponderando las observaciones en función de su cercanía
espacial a cada punto donde se está realizando la estimación.

Para estimar los residuos he ajustado el modelo GWR con el ancho de
banda óptimo encontrado y posteriormente he extraido los residuos del
modelo GWR. Con los residuos he creado la lista de pesos espaciales
basada en los vecinos más cercanos y finalmente he realizado el test de
Moran en los residuos, obteniendo un p valor muy bajo (\textless0.05),
por lo que podemos determinar que ** existe autocorrelación espacial
restante en los residuos, indicando que el modelo no ha capturado
completamente la estructura espacial**.

\begin{verbatim}
## Bandwidth: 0.1068962 CV score: NA
\end{verbatim}

\begin{verbatim}
## Bandwidth: 0.172789 CV score: NA
\end{verbatim}

\begin{verbatim}
## Bandwidth: 0.213513 CV score: NA
\end{verbatim}

\begin{verbatim}
## Bandwidth: 0.2386818 CV score: NA
\end{verbatim}

\begin{verbatim}
## Bandwidth: 0.254237 CV score: NA
\end{verbatim}

\begin{verbatim}
## Bandwidth: 0.2638506 CV score: NA
\end{verbatim}

\begin{verbatim}
## Bandwidth: 0.2697922 CV score: NA
\end{verbatim}

\begin{verbatim}
## Bandwidth: 0.2734643 CV score: NA
\end{verbatim}

\begin{verbatim}
## Bandwidth: 0.2757337 CV score: NA
\end{verbatim}

\begin{verbatim}
## Bandwidth: 0.2771364 CV score: NA
\end{verbatim}

\begin{verbatim}
## Bandwidth: 0.2780032 CV score: NA
\end{verbatim}

\begin{verbatim}
## Bandwidth: 0.278539 CV score: NA
\end{verbatim}

\begin{verbatim}
## Bandwidth: 0.2788701 CV score: NA
\end{verbatim}

\begin{verbatim}
## Bandwidth: 0.2790747 CV score: NA
\end{verbatim}

\begin{verbatim}
## Bandwidth: 0.2792012 CV score: NA
\end{verbatim}

\begin{verbatim}
## Bandwidth: 0.2792794 CV score: NA
\end{verbatim}

\begin{verbatim}
## Bandwidth: 0.2793277 CV score: NA
\end{verbatim}

\begin{verbatim}
## Bandwidth: 0.2793277 CV score: NA
\end{verbatim}

\begin{verbatim}
##           Length  Class                  Mode     
## SDF          2000 SpatialPointsDataFrame S4       
## lhat      4000000 -none-                 numeric  
## lm             11 -none-                 list     
## results        14 -none-                 list     
## bandwidth       1 -none-                 numeric  
## adapt           0 -none-                 NULL     
## hatmatrix       1 -none-                 logical  
## gweight         1 -none-                 character
## gTSS            1 -none-                 numeric  
## this.call       8 -none-                 call     
## fp.given        1 -none-                 logical  
## timings        12 -none-                 numeric
\end{verbatim}

\begin{verbatim}
## [1]  20.6359032   1.6456017 -28.8882135  11.2363254  -0.8449863  -3.6437980
\end{verbatim}

\begin{verbatim}
## 
##  Moran I test under randomisation
## 
## data:  residuos_gwr  
## weights: listw2    
## 
## Moran I statistic standard deviate = 1.485, p-value = 0.06878
## alternative hypothesis: greater
## sample estimates:
## Moran I statistic       Expectation          Variance 
##      0.0212815954     -0.0005002501      0.0002151618
\end{verbatim}

\end{document}
